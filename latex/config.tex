% Codificación predeterminada en UTF-8 (puede ser que esto sea innecesario en versiones recientes de LaTeX).
\usepackage[utf8]{inputenc}

% Elección de idioma
\usepackage[spanish]{babel}

% Margenes (Se usan las dadas dentro de las normas APA)
\usepackage[top=2.54cm, bottom=2.54cm, left=2.54cm, right=2.54cm]{geometry}

% Habilitación de interlineado
\usepackage{setspace}

% Acentuación mejorada en fuentes.
\usepackage[T1]{fontenc}

% Soporte de latino moderno romano (caracteres especiales de otro idiomas como el alemán).
\usepackage{lmodern}

% Mejora en posicionamiento de figuras e imágenes.
\usepackage{float}

% Imágenes en texto
\usepackage{wrapfig}

% Habilitación de iconos de Fontawesome.
\usepackage{fontawesome5}

% Habilitación de uso de colores.
\usepackage[dvipsnames]{xcolor}

\definecolor{codebg}{RGB}{249, 245, 215}
\definecolor{codered}{RGB}{204, 36, 29}
\definecolor{codegreen}{RGB}{152, 151, 26}
\definecolor{codeyellow}{RGB}{215, 153, 33}
\definecolor{codeblue}{RGB}{69, 133, 136}
\definecolor{codepurple}{RGB}{177, 98, 134}
\definecolor{codegray}{RGB}{124, 111, 100}
\definecolor{codefg}{RGB}{60, 56, 54}
\definecolor{backcolour}{RGB}{246,247,246}

% Soporte de manipulación de imágenes.
\usepackage{graphicx}
\graphicspath{ {assets/img/} }

% Espacio entre dos párrafos.
\usepackage{parskip}

% Caracteres de dibujo de caja de páginas de códigos antiguas.
\usepackage{pmboxdraw}

% Mejora en tablas
\usepackage{multirow}

% Propiedades y vínculos de archivos PDF
\usepackage{hyperref}
\hypersetup{
    pdftitle = {La plataforma de desarrollo GLib/GTK},
    pdfauthor = {Sébastien Wilmet, Gerson Benavides},
    pdfcreator = {TeX Live},
    pdfproducer = {TeX Live},
    pdfborder = 0 0 0
}

% Enlaces extensos en varias lineas

\usepackage{xurl}

% Ajuste de paginas
\usepackage{pagecolor}
\usepackage{afterpage}

% Listados de lenguajes de programación.
\usepackage{listings}

% Personalización de estilos generales
\lstdefinestyle{gruvbox}{
    % Estilo de texto
    basicstyle=\ttfamily\footnotesize,
    breakatwhitespace=false,         
    breaklines=true,                 
    captionpos=b,                    
    keepspaces=true,
    showspaces=false,                
    showstringspaces=false,
    showtabs=false,                  
    tabsize=2,
    %%%
    % Colores de texto
    backgroundcolor=\color{backcolour},   
    commentstyle=\color{codegray},
    keywordstyle=\color{codered},
    stringstyle=\color{codegreen},
    identifierstyle=\color{codefg},
    % frame=single,
    % numbers=left,                    
    % numbersep=5pt,
    % numberstyle=\tiny\color{codegray}
    %%%
    % Añadir soporte a caracteres especiales
    literate=
    {á}{{\'a}}1 {é}{{\'e}}1 {í}{{\'i}}1 {ó}{{\'o}}1 {ú}{{\'u}}1
    {Á}{{\'A}}1 {É}{{\'E}}1 {Í}{{\'I}}1 {Ó}{{\'O}}1 {Ú}{{\'U}}1
    {ñ}{{\~n}}1 {Ñ}{{\~N}}1 {ç}{{\c c}}1 {Ç}{{\c C}}1
    {¡}{{!`}}1 {¿}{{?`}}1
}

\lstset{style=gruvbox}

% Personalización de lenguajes
\lstdefinestyle{GLib/GTK}{language=C,
    morekeywords={
        gint, gint8, gint16, gint32, gint64,
        guint, guint8, guint16, guint32, guint64,
        gboolean,
        gchar, guchar,
        gshort, gushort,
        glong, gulong,
        gfloat,
        gdouble,
        gpointer, gconstpointer,
        gsize, gssize,
        goffset,
        GSList,
        GTree,
        GNode,
        GFunc,
        GSourceFunc
    },
}

\newcommand{\shellcmd}[1]{\texttt{#1}}

% Ajustes del libro

\newcommand{\bookversion}{0.9 alpha}

\renewcommand{\lstlistingname}{Listado}

